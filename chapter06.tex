\chapter{考察}
\label{chap:discussion}
\fancyhf{}
\rhead{\thepage}
\lhead{第\ref{chap:discussion}章 考察}
\cfoot{\thepage}

本章では,実験結果を踏まえた考察を述べる.


\section{タグ抽出構造}
オートエンコーダを用いた場合と用いない場合の違いについて?



\section{抽出された知識分類}
階層性
モジュール性
可読性と予測性の違い




\section{本手法の他MOOCsや他科目への適用可能性}
本研究で用いた,問題回答から知識分類を抽出する手法の,他のMOOCsや他科目への適用可能性について考察する.
本研究の手法は,
1) データセット作成,2) 問題と知識分類の関係行列の算出,3)知識分類の抽出と分析
の3つの要素から構成されていた.
1)のデータセット作成については,MOOCsから収集される問題回答ログデータは,MOOCsや科目によらず大規模であると考えられる.
また,3)の知識分類の抽出と分析については,知識獲得を適切に表現できる問題と知識分類の関係行列を,学習の過程で抽出できるかに依存する.
よって,本手法の他MOOCsや他科目への適用可能性は,2)問題と知識分類の関係行列の算出に依存すると考えられる.
この関係行列はDKTを拡張したモデル構造において学習されるため,
DKT自体の他MOOCsや他科目への適用可能性によって,本手法の適用可能性も検証されると考えられる.


そこで,DKTの他MOOCsや他科目における予測性能を考察する.
\cite{piech2015deep}では,本研究同様,数学に関するデータセットにおいてのみ,DKTの有効性が検証されていた.
那須野らは(参照)学習サプリのデータを使って,
算数や数学に関するデータセットと地理や歴史に関するデータセットにDKTを適用した場合,
Bayesian Knowledge Tracingからの精度向上という点では大きな差はないことを確認しており,
DKTの適用可能性は科目に依らない可能性がある.


また,
\cite{piech2015deep}を始めとする既存研究では,モデルへの入力次元には問題に割り当てられたタグが利用されており,
DKTの有効性はタグを用いた場合のみ,検証されていた.
本研究の実験では,
抽出されたタグを比較検証するために,既存のタグが存在するデータセットを用いたものの,
問題回答のみから知識タグを抽出できることは,実験結果から示されており,
このことから,既存のタグが存在しないような他MOOCsや他科目のデータに対しても,
生徒の知識獲得を予測することが可能であることを示している.


一方で,これまで検証されているのは,
特定の科目に関する問題回答ログであり,
総合的な知識レベルを問うような,複数の科目が含まれている問題回答ログへの適用可能性は示されていない.

また,利用できるデータセットは,生徒が該当のMOOCsで学習する過程で,
段階的に知識を獲得していく前提のデータのみであり,
MOOCs外での学習や,生徒ごとの能力差,事前知識などの情報に関しては,
DKTが扱うことは難しい可能性がある.


以上のような考察を踏まえると,
本手法は,DKTが分析可能な他MOOCsや他科目のデータに加え,
DKTが分析できない,事前の知識分類が存在しないデータに対しても適用できるという側面がある一方,
複数科目のデータや生徒に関する事前情報など,
現実に即した複雑な情報が多く含まれたデータに対しては,適用可能性が限定的である側面もある.
% このような例として具体的にどのようなデータが考えられるか?


本論文では,
知識獲得予測の文脈において既存の知識分類よりも適切な知識分類を,
深層学習によって抽出できることを示すために,既存の知識分類がつけられた,数学のMOOCsのデータを用いて比較検証した.
本手法が,MOOCsや科目によらず,事前の知識分類を必要とせずに,
的確な知識分類を学習できるとすれば,
様々な科目のMOOCsのデータに適用することで,
人間の知識獲得のメカニズムの解明や,
既存の教育カリキュラムの妥当性の検証,
さらには生徒一人ひとりの知識状態に最適化された教育設計が
できるようになる可能性がある.
% indivisuakizationについて弱いかも,どこに論点を置くか?




\section{今後の展望}
本研究の今後の展望について大きく2つの方針を述べる.
1つは教育学における対象データの拡大についてであり,
1つは教育学以外の分野への本手法の応用についてである.


\subsection{教育学における対象データの拡大}
まず,教育学における対象データの拡大について述べる.
対象データの拡大とは,科目や難易度の多様化,予測期間の長期化,そして複数科目の統合である.


まず,科目や難易度の多様化について述べる.
本研究では.算数や数学の問題回答ログに対して深層学習を適用し,
知識獲得予測に適した知識分類を得た.
これまでのDKTの研究成果から,算数や数学以外の教科に対しても適用できる可能性は高いが,
実際にどのような知識分類が抽出されるかは分析していない.
また,今回扱ったデータセットは,小学校から高校程度の算数や数学に関する問題回答であり,
より高度で専門的な大学レベルの学問に適用する場合についても,
どのような知識分類が抽出されるかは分析していない.
知識獲得の最適化に関する知見は,学校側からの指導や生徒自身の学習設計に活用されており,
多様な難易度や科目において知識獲得を最適化する知識構造を明らかにすることは,重要であると考えられる.


次に,予測期間の長期化について述べる.
本研究で用いたデータセットの対象期間は,1〜2年程度であった.
しかし,DKTを用いた生徒の知識獲得の予測は,
それまでの生徒の知識獲得の過程に依存しており,
できるだけ長い期間の知識獲得を分析するほうが,
より高い精度で知識獲得を予測でき,また,より適切な知識分類を抽出できる可能性は高い.


最後に,複数科目の統合について述べる.
本研究や既存研究では,特定の科目について独立に知識獲得を予測し,知識構造を分析している.
しかし,実際の生徒の学習の成長過程は,科目間で完全に独立であるとはいえず,
例えば,歴史と地理や,数学と物理などの科目間では,知識獲得の過程が密接に関係している可能性がある.
一人の生徒の,科目を横断した知識獲得過程に関する研究は,これまで報告されていないが,
複数科目を統合したデータに対して本手法を適用することで,科目内の分類や学習設計だけでなく,
包括的な学習設計や,既存の科目という分類にとらわれない,人間が獲得する知識一般に関する知見も得られる可能性がある.
% もう少しスケールダウンして,論理の飛躍抑える



\subsection{教育学以外の分野への応用}
次に,本手法の教育学以外の分野について述べる.

本手法は,Knowledge Tracingという,
教育学の,知識獲得予測タスクにおける手法だが,
より手法を一般化することで,教育学以外の分野にも応用できる可能性を秘めている.

本手法は,
生徒の時系列問題回答ログから,
回答を重ねるごとに遷移していく知識状態をモデリングし,
知識獲得の過程を適切に表現する知識分類を抽出するというものだが,

これをより一般化して捉えると,
人間の,何らかのコンテンツ集合に対する時系列行動ログから,
行動を重ねるごとに遷移していく人間の何らかの状態をモデリングし,
行動の遷移を適切に説明する分類表現を抽出しているといえる.

教育学の知識獲得の分野においては,
このコンテンツ集合に対する行動が生徒の問題回答であり,
問題回答により遷移する生徒の知識状態をモデリングしているが,
これと同様のことは,教育学に限らず行える可能性がある.

例えば,消費者が商品を購買するログを分析することで,
消費者の消費嗜好が遷移する過程をモデリングし,
従来の商品分類と異なる,消費者の消費嗜好の遷移を反映した分類を抽出することが可能になる.
教育学では,問題回答の正誤と知識の間の特殊な関係性をモデル設計に反映しているように,
手法を適用する領域によって調整は必要であるが,
コンテンツに対する行動の時系列性を反映した分類を抽出できる可能性は高く,
様々な領域で,重要な意味を持つ新たな知見を得られると考えられる.




\vvspace
以上,考察について述べた.
次章では,結論を述べる.
