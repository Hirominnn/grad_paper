% 問題意識追加

% 背景1;オンライン学習サービスの普及によるビッグデータ蓄積
近年,オンライン教育サービスの普及に伴い,
教育におけるビッグデータの活用が進んでいる.

オンライン教育サービスは,
既存の学校教育と異なる,新たな学習機会を提供することに加え,
サービスを利用する生徒の学習行動ログを収集することが可能なため,
これまで困難であった,大規模な学習効果分析が可能になった.


% 背景2:深層学習の高まり
一方,近年,教育に限らない多くの研究領域で,深層学習が注目されている.
人間の脳の構造を模した多層のニューラルネットワーク構造によって,
従来の機械学習では難しかった,データの複雑な意味表現を抽出することができ,
多様な領域で大きな進展がもたらされている.

こうした深層学習の技術を,
オンライン教育サービスに蓄積された大規模な学習行動ログに対して適用することで,
これまで困難だった,人間の複雑な知識構造や学習効果の分析が可能になった.

特に,生徒の学習効果を最適化する目的で古くから行われてきた,
生徒の知識獲得を予測する研究は,
オンライン学習サービスにおける生徒の学習効率向上に直接的に寄与することに加え,
深層学習を適用することによって,高い精度での予測が可能な上,知識間のネットワーク構造を抽出できることが報告されており,
今,大きな注目を集めている研究分野の一つである.

% 人間を超えるor人間で判断できない部分の強みとしての要素?

% 本論文の目的
本研究では,生徒の学習行動ログに対して,深層学習を適用し,生徒の知識獲得を予測する.
既存の研究では,事前に専門家が作成した知識分類を用いて,知識獲得の予測が行われていたが,
本論文では,人間が作成した知識分類は,現実の生徒の知識獲得を説明する上で最適化されたものではない,という仮説に立ち,
深層学習モデルによって生徒の知識獲得予測を行う過程で,
モデル自身に新たな知識分類を自動で抽出させることで,
より最適化された知識分類を獲得することを目的とする.

実験の結果,
深層学習が抽出した知識分類を用いることで,
人間が作成した既存の知識分類を用いる場合より高い精度で
知識獲得を予測できることが検証された.

また,抽出された知識分類を分析した結果,
既存の分類を横断するような,複数の分野の基礎となっている知識表現や,
これまで一つとされていた分野をより階層的に分解するような,難易度を表す知識表現などが獲得されていることを確認した.

% 考察
これらの検証結果は,
知識獲得を予測する上では,
人間が定義した知識分類よりも優れた知識分類を,
深層学習が獲得したことを示しており,
この知識分類を活用することにより,
より最適化された学習プロセスを設計できる可能性を示唆している.

さらに,
研究の拡張として,
本研究で用いた分析手法の教育学での適用可能性や,
教育学以外の分野への応用可能性を考察した.


% まとめ
本研究は.
オンライン学習サービスの普及や,教育分野におけるビッグデータ分析の活発化,深層学習の躍進など,
ここ数年の多様な領域の進展によって初めて可能になったものである.
本研究が,既存の教育システムの再構築,そして人間の学習や知識の解明につながると信じている.


