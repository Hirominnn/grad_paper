\chapter{結論}
\label{chap:conlusion}
\fancyhf{}
\rhead{\thepage}
\lhead{第\ref{chap:conlusion}章 結論}
\cfoot{\thepage}


本論文では,
既存の知識獲得予測における問題として,
人間が作成した知識分類を利用していることを指摘し,
深層学習を適用することによって,
知識獲得の予測において最適化された知識分類を抽出できることを検証した

また,抽出された知識分類を定性的に分析することにより,
分野の内容的な関係性と,分類ごとの情報量を最適に分配することが,知識獲得の予測性の向上に寄与することがわかった.
\vvspace

本研究の成果を実際の教育現場で活用する例の一つとして,
教材推薦システムへの適用を論じた.

また,本研究の分析手法の他の科目やオンライン教育サービスへの適用可能性について議論し,
科目や既存の知識分類の有無によらずに最適な知識分類を学習して知識獲得を予測できる可能性があること,
および,
複数科目を統合した知識獲得や,大学水準の知識獲得に関する分析については,
検証実験を行う必要が有ることを述べた.
\vvspace


さらに,本研究の拡張として,
より良質な知識分類を学習するために活用できる最新の深層学習技術や,
適用対象データの拡張について述べ,
また,より一般性を高めた議論として,
教育学以外の領域においても本手法が適用できる可能性に触れ,
人間行動に関する多様な知見を発見できる可能性を論じた.
\vvspace

本研究は.
教育と情報技術の融合の進展やオンライン教育サービスの普及,教育分野における大規模分析の活発化や深層学習の躍進など,
ここ数年の多様な領域の進展によって初めて可能になったものである.
本研究が,既存の学問体系の再構築,そして人間の学習や知識の解明につながると信じている.
