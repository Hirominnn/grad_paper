\chapter{結論}
\label{chap:conlusion}
\fancyhf{}
\rhead{\thepage}
\lhead{第\ref{chap:conlusion}章 結論}
\cfoot{\thepage}


本論文では,
既存の知識獲得予測における問題として,
人間が作成した知識分類を利用していることを指摘し,
深層学習を適用することによって,
知識獲得の予測において最適化された知識分類を抽出できることを検証した

また,抽出された知識分類を定性的に分析することにより,xxxという知見を得た.
\vvspace

本研究の成果を実際の教育現場で活用する例の一つとして,
教材支援システムへの適用を論じた.

また,本研究の分析手法の他の科目やオンライン教育サービスへの適用可能性について議論し,
科目や既存の知識分類の有無によらずに最適な知識分類を学習して知識獲得を予測できる可能性があること,
および,
複数科目を統合した知識獲得や,大学水準の知識獲得に関する分析については,
検証実験を行う必要が有ることを述べた.
\vvspace


さらに,本研究の拡張として,
適用対象の拡張という点で
対象データの多様化,対象期間の長期化,複数科目の統合,の3つの拡張を述べ,
また,より一般性を高めることで,
教育学以外の領域においても本手法が適用できる可能性に触れ,
人間行動に関するの多様な知見を発見できる可能性を論じた.
\vvspace

本研究は.
オンライン教育サービスの普及や,教育分野における大規模分析の活発化,深層学習の躍進など,
ここ数年の多様な領域の進展によって初めて可能になったものである.
本研究が,既存の学問体系の再構築,そして人間の学習や知識の解明につながると信じている.
