\chapter{結論}
\label{chap:conlusion}
\fancyhf{}
\rhead{\thepage}
\lhead{第\ref{chap:conlusion}章 結論}
\cfoot{\thepage}


本論文では,
主に心理学の研究領域で議論されていた宣言的知識と手続き的知識の獲得における知識構造の違いを定量的に分析し,
知識獲得における宣言的知識の知識構造は手続き的知識の知識構造と比べてよりモジュール性が高く,
逆に,手続き的知識の知識構造は宣言的知識の知識構造と比べてより階層性が高いことを検証した.
\vvspace

検証過程で得られた知見に基づいて,
本研究で用いた知識構造の分析手法の他MOOCsへの適用可能性について議論し,
講座の科目や問題へのタグの有無に依らず知識構造を分析できる可能性があること,
および,
特に,大学水準の難易度を扱うMOOCsについては検証実験を行う必要があることを指摘した.
\vvspace

また,本研究の拡張として,
適用対象の拡張という点で
対象データの多様化,結合,長期化の3つの拡張を述べた.
また,
知識間関係抽出法であるDeep Knowledge Tracingの拡張という点で,
カリキュラム学習による拡張,および,
スキルタグ自動抽出による拡張の2つについて述べた.
\vvspace


本研究は
MOOCsの登場,ネットワーク分析の発展,深層学習の躍進等,
ここ数年の幅広い領域のさまざまな成果によって,初めてその実施が可能になったものである.
本研究が,人間の学習や知能の解明の一助になると信じている.






