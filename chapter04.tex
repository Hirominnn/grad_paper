\chapter{データセット}
\label{chap:dataset}
\fancyhf{}
\rhead{\thepage}
\lhead{第\ref{chap:dataset}章 データセット}
\cfoot{\thepage}


本章では,
実験で用いるデータセットについて述べる.
% 課程のレベルと,日本と比較するとどの粒度なのか

% 数学であることを述べる
主に用いるデータセットは,世界最大級のオンライン学習プラットフォームの一つである
「ASSISTments\footnote{\url{https://www.assIstments.org/}}」における,
2009年から2010年にかけての,数学の問題に対する生徒の問題回答ログを用いる.

また,手法を比較検証するために,
コンピューターサイエンス分野で最も古く権威ある競技会の一つである
「KDDCup(Knowledge Discovery and Data Mining Cup)\footnote{\url{http://www.kdd.org/kdd-cup}}」の,2010年の競技会において利用されたデータセットも利用する.
このデータセットも,ASSISTments同様,数学の問題に対する生徒の問題回答ログである.

% データセットの項目で,数学である理由を述べておく
いずれのデータセットも,前章で述べた,データセットの要件の一つである,既存の知識分類を有するという要件を満たしている.


以下では,各データセットについて概説した後,
本研究に適用するためのデータの抽出方法について述べる.
% データの概観?


\section{ASSISTments 2009-2010}
% ASSISTmentsに関する解説
ASSISTmentsのサービスについて概説した後,
本研究で用いるデータセットについて説明する.
その際,本研究に適用する際に問題となるデータの性質に言及した上で,
その問題点を解消するためのデータの抽出方法を述べる.

\subsection{ASSISTmentsのサービス}
ASSISTmentsはITS(Intelligent Tutoring System)の一つで,

オンライン学習支援プラットフォーム
数学

ASSISTmentsにはxxxなどのさまざまなデータが存在し,教育科学において広く利用されるデータセットである.
ASSISTments自身が,データを利用した教育科学の研究を奨励しており,
ベンチマークとして利用されるだけでなく,様々な研究成果を生み出している.(参照)


\subsection{対象データセット}
%データセット
その中でも,本研究で用いるデータセットは,「ASSISTments 2009-2010」と呼ばれる,
ASSISTmentsにおける,生徒の2009年から2010年の間の問題回答ログである.
「ASSISTments 2009-2010」には「skill\_builder」と「non\_skill\_builder」のデータセットが含まれている.
前者は,アメリカでは「formative assesment」と呼ばれる,生徒に知識を段階的に身につかせることを目的にしており,
ある知識を問う問題を生徒が連続で正答できた場合に,その生徒が該当の知識を習得したものとみなし,次に進ませるというものである.
後者は,アメリカでは「summative assesment」と呼ばれ,生徒がそれまで学んできたことを正しく身につけられているかを確認することを目的にしており,
各知識を問う問題を,まとめて生徒に課すものである.日本の教育現場でいえば,期末試験に近いものといえる.

このような性質から,前者のデータセットの方が,生徒の知識獲得の過程を観察する上で適しているため,
DKT(参照)を始めとするKnowledge Tracingに関する多くの研究で利用されるデータセットであり,
本研究でも同様に,前者のデータセットを用いる.

% 概観
「skill\_builder」のデータセットには,4217人の生徒の,124のスキルタグが紐づく26688の問題に対する,401756の回答ログが含まれている.
なお,このデータセットは,行の重複などによって不備が指摘されたためxxx年に訂正版が提供されているおり,
それ以前の研究結果は信憑性に疑問があるとされている.(going deeper参照)本研究では,訂正後のデータセットを用いている.


\subsection{データの抽出}
%抽出条件
本データセットは,本研究に適用する上で幾つかの問題を抱えている.

まず,問題の中には複数のスキルタグが紐付いているものがあるが,そうした問題が回答された場合には,
スキルタグの数だけ,ログが別々に作成されている.
これは,見かけ上のログ数が,実際に回答された回数より多くなっているだけでなく,
同時に回答された問題やスキルタグを,別々に回答されたとみなされる危険性があり,
この前提を考慮しないKnowledge Tracingは不適切であることが指摘されている.(参照)
このため,重複している行を一つにまとめる作業が必要である.

次に,既存の知識分類であるスキルタグも,存在はするものの,名前が割り当てられていないものが存在し,
これらは抽出された知識分類と比較することが不可能なため,除外する必要がある.

また,本データセットは,全体で見ると十分大規模であるといえるが,
個別の生徒や問題に関して言えば,ほとんど回答していない生徒や,ほとんど回答されていない問題が,
全体に対して大きな割合を占めている.(図参照)
十分なログ数を保有しない生徒や問題は,大規模データから深層学習によって知識構造を抽出するという本研究の目的を満たさないため,
一定の閾値によってデータセットを切り分けることによって,適切なデータを抽出する必要がある.


以上より,本研究では,元のデータセットから,以下の方法で分析対象とするデータを抽出している.
1)同時回答を意味する重複行を一つにまとめる.
2)名前が割り当てられているスキルタグを持つ問題に関するログのみを抽出する.
3)2)のうち,最低50回以上回答されている問題に関するログのみを抽出する.
4)3)に含まれる問題を,最低2回以上回答している生徒に関するログを抽出する.

なお,3)の50回以上という具体的な数字は,深層学習として有意な結果が得られるログ数として,実験的に得たものであり,最適な数値であるという保証をするものではない.
結果的に,2809人の,31のスキルタグが紐づく967の問題に対する,66707の回答ログが分析対象である.




\section{KDDCup 2010}
KDDCupについて概説した後,
本研究で用いるデータセットについて説明する.
その際,ASSISTments同様,本研究に適用する際に問題となるデータの性質に言及した上で,
その問題点を解消するためのデータの抽出方法を述べる.


\subsection{KDDCup}
KDDCup(Knowledge Discovery and Data Mining Cup)は、
100以上の国にまたがり、10万人を超える会員を持つコンピューターサイエンス分野の学会である「ACM(the Association for Computing Machinery)」の分科会である
「SIGKDD(Special Interest Group on Knowledge Discovery and Data Mining)」が毎年開催する競技会であり、
この分野で最も古く権威のある競技会の一つである.


\subsection{対象データセット}
本研究では,2010年に開催されたKDDCupの内の一つの競技会である「Educational Data Mining Challenge」で使用された,
「Bridge to Algebra 2006-2007」というデータセットを用いる.
これは,オンライン学習プラットフォーム「Carnegie Learning\footnote{\url{https://www.carnegielearning.com/}}」が提供する
オンライン学習支援システム「Cognitive Tutor」における,2006年から2007年の間の,数学の問題に対する生徒の問題回答ログである.
「Cognitive Tutor」は認知モデル(参照)を用いた,自動xxxである.
ASSISTmentsが生徒の宿題xxxなのに対し,Cognitive Tutorではxxxと言われている.
% 概観,problemとstep, kcの関係述べる

% 提供されているデータの中からbridge...を選択する理由
このデータセットには,1146人の生徒の,500?のスキルタグが紐づく19186の問題と19766のステップに対する,3679199の回答ログが含まれている.


\subsection{データの抽出}
本データセットも,「ASSISTments 2009-2010」同様に,本研究に適用する上で幾つかの問題を抱えている.

まず,problemやstepという粒度が存在する本データセットにおいて,何を一回の「問題回答」と見なしてデータを作成するかが問題となる.
「Cognitive Tutor」では,一つのproblem内に複数のstepが用意されており,各stepについて逐次回答し,正答できるまで取り組み,
正答できた場合に次のstepに進むことができるようになっている.
そのため,生徒の知識獲得の推移を観察する上では,一つ一つのstepに注目することが適切だといえる.
よって,本データセットでは,problemとstepの組み合わせに対する回答を「問題回答」と見なし,データを抽出する.

次に,既存の知識分類であるknowledge conceptについても,存在はするものの,名前が割り当てられていないものが存在し,
これらは抽出された知識分類と比較することが不可能なため,除外する必要がある.

また,本データセットは,全体で見ると十分大規模であるといえるが,
個別の生徒や問題に関して言えば,ほとんど回答していない生徒や,ほとんど回答されていない問題が,
全体に対して大きな割合を占めている.(図参照)
十分なログ数を保有しない生徒や問題は,大規模データから深層学習によって知識構造を抽出するという本研究の目的を満たさないため,
一定の閾値によってデータセットを切り分けることによって,適切なデータを抽出する必要がある.


以上より,本研究では,元のデータセットから,以下の方法で分析対象とするデータを抽出している.
1)問題(problem)とステップ(step)の組み合わせを一つの問題回答とみなす,
2)名前が割り当てられているスキルタグを持つ問題に関するログのみを抽出する.
3)2)のうち,最低100回以上回答されている問題に関するログのみを抽出する.
4)3)に含まれる問題を,最低2回以上回答している生徒に関するログを抽出する.

なお,3)の100回以上という具体的な数字は,深層学習として有意な結果が得られるログ数として,実験的に得たものであり,最適な数値であるという保証をするものではない.
結果的に,1137人の,204のスキルタグが紐づく3495の問題に対する,613976の回答ログが分析対象である.



\section{データセットの概観}






\section{データセットの概観}
\begin{table}[!htb]
\caption{11データセットの統計量}
\label{tab:datasets-overview}
\begin{center}
\centerline{
{
\begin{tabular}{cc|rrrrr}\hline\hline
\multirow{2}{*}{学年}&	\multirow{2}{*}{科目}	&	\multirow{2}{*}{ユーザ数}	&	\multirow{2}{*}{問題数}		&	\multirow{2}{*}{回答ログ数}		&	\multirow{2}{*}{\shortstack{回答ログ数\\ $\div$ユーザ数}}	&		\multirow{2}{*}{\shortstack{回答ログ数\\ $\div$問題数}}		\\
				&								&								&								&									&				&			\\\hline
\multirow{2}{*}{小学4年}&		社会			&	3,045						&	76							&	227,409			&	75							&	2,992	\\	
				&				算数			&	4,318						&	182							&	505,917			&	117							&	2,780	\\\hdashline	
\multirow{2}{*}{小学5年}&		社会			&	2,833						&	197							&	388,521			&	137							&	1,972	\\	
				&				算数			&	3,380						&	257							&	411,957			&	122							&	1,603	\\\hdashline	
\multirow{2}{*}{小学6年}	&	社会			&	2,891						&	202							&	434,324			&	150							&	2,150	\\	
				&				算数			&	3,225						&	245							&	395,276			&	123							&	1,613	\\\hdashline	
中学1年			&				数学			&	7,137						&	365							&	659,237			&	92							&	1,806	\\	
中学2年			&				数学			&	3,931						&	278							&	238,241			&	61							&	857	 	\\   
中学3年			&				数学			&	2,667						&	343							&	177,295			&	66							&	517	 	\\\hdashline   
\multirow{2}{*}{中学}&			地理			&	6,499						&	308							&	660,882			&	102							&	2,146	\\	
				&				歴史			&	6,381						&	364							&	853,419			&	134							&	2,345	\\	
\hline\hline
\end{tabular}
}
}
\end{center}
\end{table}

% データセットの概観


\vvspace
以上,データセットについて述べた.
次章では,実験について述べる.

