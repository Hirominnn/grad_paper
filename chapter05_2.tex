\chapter{実験}
\label{chap:result}
\fancyhf{}
\rhead{\thepage}
\lhead{第\ref{chap:result}章 実験}
\cfoot{\thepage}


本章では,実験について述べる.

まず,実験設定について述べ,
その後,実験結果について述べる.


\section{実験設定}

本研究の実験は,大きく以下の3つのブロックに分けられる.
\begin{enumerate}
	\item 知識分類学習モデルによる知識分類の学習
	\item 学習された知識分類の知識獲得予測の性能に関する検証
	\item 学習された知識分類の性質に関する比較分析
\end{enumerate}
以下では,順に,実験設定について述べる.


\subsection{知識分類学習モデルによる知識分類の学習}
\label{sec:section}
生徒の問題回答ログに対し,図\ref{fig:model}に表される知識分類学習モデルを適用し,回答正誤の予測性において最適な知識分類表現を抽出する.
知識タグ空間の次元数は,既存の知識タグの次元数と統一し,
「ASSISTments 2009-2010」では56,
「Bridge to Algebra 2006-2007」では193とした.
実際のモデルにおいては,正答ベクトルと誤答ベクトルを分けてユニットを作るため,
それぞれ2倍のユニット数で表現されている.

ハイパーパラメータについては,
%対象データが多いため学習コストの削減を狙いRNNの部分にはGRNNを用いる.
学習率の初期値を$200$,
モーメントを$0.98$,
1エポックごとに,
減衰率$0.99$として学習率を最小学習率$10$まで減衰させる.
また,勾配のノルムの最大値を$0.00001$として\cite{pascanu2013difficulty}に従い勾配に制約を設けた.
dropoutは\cite{piech2015deep}と同様に${\bf y}_t$の方向にのみかけ,
dropout率は$0.5$とした.
隠れ層のユニット数は$400$として,
各重み行列の初期化は\cite{glorot2010understanding}にしたがった.
時系列方向の誤差逆伝搬は最長で$200$まで伝搬するように制約を設けた.

これらのハイパーパラメータは実験的に高い予測性能を発揮したため設定しており,
網羅的に探索したわけではない.
通常,深層学習の手法はハイパーパラメータの数が非常に大きく,また,
計算コストが大きいため大規模な探索は行えない.
Grid SearchやRandom Search\cite{bergstra2012random}といった探索手法が提案されているが,
専門家が手で調整した方が優れていることが報告されている\cite{larochelle2007empirical, bergstra2012random}.

最適化手法は,
式\ref{eq:prediction_entropy}で表される問題回答予測に関する誤差関数$L_p$と,
式\ref{eq:autoencoder_loss}で表される問題空間と知識タグ空間の最高性誤差を表す誤差関数$L_r$の和である
$L$(式\ref{eq:total_loss})を目的関数として最小化するものである.
学習時は\cite{piech2015deep}と同様にミニバッチごとに確率的勾配降下法で目的関数を最小化する.
評価指標はAUCスコアを採用する.

2つのデータセットいずれにおいても,
訓練:検証:テスト = 8:1:1となるようにユーザを分け,
訓練ユーザのデータでモデルを構築し,
検証ユーザのデータでハイパーパラメータを調整し, 
検証ユーザのデータで精度が最も高かったモデルを
テストユーザのデータに適用し当該モデルの最終的な精度とした.

実装には
Theanoを用いた\cite{bergstra+al:2010-scipy,Bastien-Theano-2012}.
Theanoは多次元行列を含む数学的表現の定義や計算,最適化を効率的に行えるPythonのライブラリで,
深層学習の研究ではよく利用される.


\subsection{学習された知識分類の知識獲得予測の性能に関する検証}

\ref{sec:section}で得られた問題空間から知識タグ空間への写像行列$P$を知識分類表現と見なしてDKTを行い,
既存の知識分類を用いた場合との精度の比較を行う.
また,本手法で抽出される知識分類と既存の知識分類の差分を明確にするため,
以下の方法で作成された知識分類を用いた場合とも比較を行う.
\begin{itemize}
\item DKTを用いず,一般的な事前学習のAutoencoderによって作成された知識分類 \label{c1}
\item DKTを用いるが,再構成誤差を目的関数に導入しない,一般的な埋め込み(Embedding)によって作成された知識分類 \label{c2}
\end{itemize}

%まず,写像行列の離散化は,いずれのデータセットにおいても,
%各問題に対して最も関係性の強い知識タグのみを1とし,他を0にする方法が,
%最も高い精度で知識獲得を予測できることを確認したため,
%この手法を用いる.

ハイパーパラメータについては,
%対象データが多いため学習コストの削減を狙いRNNの部分にはGRNNを用いる.
学習率の初期値を$200$,
モーメントを$0.98$,
1エポックごとに,
減衰率$0.99$として学習率を最小学習率$10$まで減衰させる.
また,勾配のノルムの最大値を$0.00001$として\cite{pascanu2013difficulty}に従い勾配に制約を設けた.
dropoutは\cite{piech2015deep}と同様に${\bf y}_t$の方向にのみかけ,
dropout率は$0.5$とした.
隠れ層のユニット数は$400$として,
各重み行列の初期化は\cite{glorot2010understanding}にしたがった.
時系列方向の誤差逆伝搬は最長で$200$まで伝搬するように制約を設けた.

最適化手法は,
一般的なDKTと同じく,
式\ref{eq:prediction_entropy}で表される回答正誤予測に関する誤差関数$L_p$
を目的関数として最小化するものである.
学習時は\cite{piech2015deep}と同様にミニバッチごとに確率的勾配降下法で目的関数を最小化する.
評価指標はAUCスコアを採用する.

\ref{sec:section}と同様に,
2つのデータセットいずれにおいても,
訓練:検証:テスト = 8:1:1となるようにユーザを分け,
実装にはTheanoを用いた.


\subsection{学習された知識分類の性質に関する比較分析}

\ref{sec:section}で抽出された知識分類を既存の知識分類を比較分析することで,
その性質を検証する.

まず,各知識タグが回答ログに出現する頻度の分布や,紐づく問題数の分布に着目し,
知識獲得予測の精度を向上させる要因を,データ構造の側面から分析する.
また,既存タグから成る知識間影響ネットワークに抽出タグを配置して可視化することで,
既存タグが抽出タグによってどのように再配置され,どのようなネットワーク構造となったのかを分析する.




以上,実験設定について述べた.



\section{実験結果}
実験結果について述べる.
まず,各手法によって作成された知識分類についての知識獲得予測における予測性能を比較し,
いずれのデータセットにおいても,
提案手法によって学習された知識分類表現を利用することで,
最も良い精度で予測が可能になっていることを定量的に確認する.

さらに,学習された知識分類を,既存の知識分類と比較することにより,
その性質を定性的に分析する.



\subsection{各知識分類の知識獲得予測における予測性能}

%\begin{table}[!htb]
\begin{table}[t]
\caption{各知識分類の知識獲得予測における予測性能}
\label{tab:result1}
\begin{center}
\centerline{
{
\begin{tabular}{c|rrrrrr}\hline
\multirow{3}{*}{データセット}	&	\multicolumn{5}{c}{AUC}\\\cline{2-6}
 							&	\multirow{2}{*}{既存タグ(marginal)} 	& \multirow{2}{*}{事前学習タグ} 	&	& \multicolumn{2}{c}{深層学習タグ}\\\\cline{5-6}
							&										& 								&	& Embedding 	& AutoEncoder \\\hline 			 
ASSISTments 2009-2010  		&				0.72 (0.61) 			& 			0.67 				& 	& 0.?? 			& {\bf 0.79}\\
Bridge to Algebra 2006-2007 &				0.79 (0.70) 			& 			0.??			 	& 	& 0.?? 			& {\bf 0.??}\\
\hline
\end{tabular}
}
}
\end{center}
\end{table}


ベースラインとなる既存の知識タグ(既存タグ)と,
DKTを用いず,一般的な事前学習のAutoencoderによって作成された知識タグ(事前学習タグ),
DKTを用いるが,再構成誤差を目的関数に導入しない,一般的な埋め込み(Embedding)によって作成された知識タグ(深層学習タグ[Embedding]),
知識分類学習モデルから作成された知識タグ(深層学習タグ[Autoencoder])
をそれぞれDeep Knowledge Tracingに適用した結果を表\ref{tab:result1}に示す.
marginalは各問題についてそれぞれ正解の周辺確率を予測結果とするものである.
\cite{piech2015deep}にも記載されていたため,本稿でも同様にベースラインの参考として記載した.
また,
値が大きい箇所は太字で記載した.


いずれのデータセットにおいても,
提案手法である「深層学習タグ[AutoEncoder]」が,最も高いAUCを記録した.


\subsection{学習された知識分類の可視化と概観}
最も良い予測性能を発揮した「深層学習タグ(AutoEncoder)」を可視化し,その概要を明らかにする.
(ネットワークや対応関係と絡めて執筆予定)
一覧 
タグ中心のネットワークとスキル中心のネットワーク比較 
2つをまとめて比較 

\subsection{学習された知識分類の比較分析}
抽出された知識タグを既存の知識タグと比較することにより,
抽出された知識タグの性質を定性的に確認する.

まず,各知識タグが回答ログに出現する頻度の分布や,紐づく問題数の分布に着目し,
知識獲得予測の精度を向上させる要因を,データの構造から分析した.(図xx)
図より,既存タグは各指標について分散が大きい一方で,
抽出タグは各指標について分散が小さく,特定の値の周辺に集中している.
この分布の違いと予測精度の関係性についても,次章で考察する.

次に,
既存のDKTの手法に基づき作成された既存タグの知識間ネットワークに対し,
既存タグと抽出タグの共起行列からTD-IDF値を求めて作成した行列を元にノードとエッジを追加したネットワーク図を図xxに示す.
図より,
表より,複数の抽出タグにおいて特徴的と判断されている既存タグが存在する一方,
特定の抽出タグ内にまとめられ,他のタグでは特徴的でないと判断されている既存タグも存在する.
この特徴付けにより,どのような性質のタグが抽出されているかについては,次章で考察する.



\vvspace
以上,実験について述べた.
次章では,考察について述べる.



